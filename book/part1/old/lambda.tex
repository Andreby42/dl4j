% !Mode:: "TeX:UTF-8"
\chapter{Lambda表达式}
\label{chap:java_lambda}

Lambda表达式是Java8引入的一个重要新特性,允许你通过表达式来代替函数。
用起来和方法一样,有一个参数列表和使用这些参数的代码。
Lambda对集合做了必要扩展,使用流(stream)遍历和操作数据。
先看几个例子:\vspace{0.3cm}

\begin{lstlisting}[language=Java]
	// 1. 无参数, 返回值为7
	IntSupplier fn = () -> 7;
	
	// 2. 接收一个参数(数字类型),返回它的2倍
	IntFunction<String> fx = x -> (2 * x) + "$";
	
	// 3. 接受2个参数(数字),并返回他们的乘积
	IntBinaryOperator f_xy_mul = (x, y) -> x * y;
	
	// 4. 接收2个int型整数,返回他们的和
	IntBinaryOperator f_xy_add = (int x, int y) -> x + y;
	
	// 5. 接受一个 string 对象,并在控制台打印,不返回任何值(看起来像是返回void)
	Consumer<String> print = (String s) -> System.out.print(s);
	
	int n = fn.getAsInt();
	String d = fx.apply(3);
	int mul = f_xy_mul.applyAsInt(2, 3);
	int add = f_xy_add.applyAsInt(2, 4);
	print.accept(String.format("%d, %s, %d, %d", n , d, mul, add));

	\\输出: 7, 6$, 6, 6
\end{lstlisting}

\section{函数分类}

$\lambda$表达式由来已久,并不是Java提出的概念,在计算机诞生之初就出现了。
这种$\lambda$表达式不仅自然,而且短小精悍。
设计之初,原本$\lambda$表达式是线程安全的,
因它不会产生副作用(影响外部的值,side effect)。

\begin{longtable}{|p{4cm}|p{9cm}|}
	分类&说明\\
	\hline
	Function
	\begin{itemize}
		\small
		\setlength{\itemindent}{-1.2em}
		\setlength{\itemsep}{0pt}
		\item R apply(T t...);
		\item applyAsInt(double);
		\item applyAsLong(double);
	\end{itemize} & \emph{运算,有入参有返回值}
	\begin{enumerate}
		\setlength{\itemsep}{0pt}
		\small
		\item BiFunction<T, U, R> 参数T和U类型,返回R类型的值
		\item DoubleFunction<R> 返回R类型的值
		\item DoubleToIntFunction 类似于类型转换
		\item DoubleToLongFunction 类似于类型转换
		\item IntFunction<R> 一个整数参数返回R类型的值
		\item IntToDoubleFunction 类似于类型转换
		\item IntToLongFunction 类似于类型转换
		\item LongFunction<R> 一个长整数返回R类型的值
		\item LongToDoubleFunction 类似于类型转换
		\item LongToIntFunction 类似于类型转换
		\item ToDoubleBiFunction<T, U> 接收T和U类型参数返回实数
		\item ToDoubleFunction<T> 类似于强制转换
		\item ToIntBiFunction<T, U> 收T和U类型参数返回整数
		\item ToIntFunction<T> 类似于强制转换
		\item ToLongBiFunction<T, U> 收T和U类型参数返回长整数
		\item ToLongFunction<T> 类似于强制转换
	\end{enumerate} \\
	\hline
	Consumer
	\begin{itemize}
		\small
		\setlength{\itemindent}{-1.2em}
		\setlength{\itemsep}{0pt}
		\item accept(T t, U u);
		\item andThen(after);
	\end{itemize} &
	\emph{有参无返回值}
	\begin{enumerate}
		\setlength{\itemsep}{0pt}
		\small
		\item Consumer 一个参数无返回值
		\item BiConsumer 两个参数无返回值
		\item DoubleConsumer 实数的特例
		\item IntConsumer 整数特例
		\item LongConsumer 长整数特例
		\item ObjDoubleConsumer 一个Obj和实数
		\item ObjIntConsumer 一个Obj和整数
		\item ObjLongConsumer 一个Obj和长整数
	\end{enumerate} \\
	\hline
	Operator
	\begin{itemize}
		\small
		\setlength{\itemindent}{-1.2em}
		\setlength{\itemsep}{0pt}
		\item Function接口;
		\item minBy(Comparator);
		\item maxBy(Comparator);
		\item compose();
		\item andThen();
		\item identity();
	\end{itemize} & \emph{操作,Function的特殊类别}
	\begin{enumerate}
		\setlength{\itemsep}{0pt}
		\small
		\item BinaryOperator<T> 同BiFunction<T,T,T>
		\item DoubleBinaryOperator 同BiFunction<T,T,T>
		\item DoubleUnaryOperator 单目运算
	\end{enumerate} \\
	\hline
	Predicate
	\begin{itemize}
		\small
		\setlength{\itemindent}{-1.2em}
		\setlength{\itemsep}{0pt}
		\item test(value);
		\item and (value);
		\item negate();
		\item or(other);
	\end{itemize} & \emph{谓词,用于真假判断}
	\begin{enumerate}
		\setlength{\itemsep}{0pt}
		\small
		\item BiPredicate<T, U> 入参类型T和U条件检查
		\item DoublePredicate 入参是实数类型
		\item IntPredicate 入参是整数类型
		\item LongPredicate 入参是长整数类型
		\item Predicate 入参是T类型
	\end{enumerate} \\
	\hline
	Supplier 
	\begin{itemize}
		\small
		\setlength{\itemindent}{-1.2em}
		\setlength{\itemsep}{0pt}
		\item T get();
		\item int getAsInt);
		\item double getAsDouble();
		\item long getAsLong();
	\end{itemize} & \emph{产生器,无入参有返回值}
	\begin{enumerate}
		\setlength{\itemsep}{0pt}
		\small
		\item Supplier 返回一个值
		\item BooleanSupplier 返回一个布尔值
		\item DoubleSupplier 返回一个实数值
		\item IntSupplier 返回一个整数值
	\end{enumerate} \\
	\hline
\end{longtable}
\noindent
除了以上已定义的$\lambda$表达式,你也可以自定义。
实际上,$\lambda$表达式都是用接口(interface)定义的。
然而在使用的时候,函数名又被省略了,所以其中存在某种对应关系。
那又怎么消除二义性的呢?这就要求接口中只能有一个未知函数。
就像这样:

\begin{lstlisting}[language=java]
	/// 定义Ixy接口
	interface Ixy {
    int add(int x, int y);
    default int sub(int x, int y) {return x-y;}
	}
  /// 定义xy变量
	Ixy xy = (a, b)-> a+b;
	int sum = xy.add(1,2);
	int sub = xy.sub(1,2);
	System.out.printf("sum=%d, sub=%d", sum, sub);
\end{lstlisting}
\noindent
注意\lstinline{default}的特殊用法,代表的是该函数接口有默认实现,
可定义多个default接口。

你也许发现了,$\lambda$表达式根本不关心关联的函数叫什么名字,只关注
名字以外的参数和类型是否一致。
\begin{lstlisting}[language=java]
	// 借用Math类的addExact
	Ixy xy = Math::addExact;
	int sum = xy.add(1,2);
	int sub = xy.sub(1,2);
	// 借用System.out::printf
	Consumer<String> print = System.out::printf;
	print.accept(String.format("sum=%d, sub=%d", sum, sub));
\end{lstlisting}

\section{应用示例}
使用$\lambda$表达式可以极大地简化代码,也使代码更加容易理解。
但$\lambda$表达式不要太复杂,建议功能单一。
接下来,将针对每一种$\lambda$表达式,举例二三。

\emph{1.过滤数组的例子}
\begin{lstlisting}[language=java]
	int[] ids = {0,1,2,3,4,5,6,7,8,9};

	Consumer<String> print = System.out::print;
	Consumer<String> println = System.out::println;
	IntPredicate isOdd = (n) -> n%2==1;

	IntConsumer c1 = (n)-> {
			if (isOdd.test(n))print.accept("单数");
			else print.accept("双数");
	};

	IntConsumer c2 = (n)-> {
			if (isOdd.test(n))println.accept("->"+n);
			else println.accept("~>"+n);
	};

	IntStream.of(ids).forEach(c1.andThen(c2));
\end{lstlisting}

这段过滤数组的例子,输出是应该怎样的呢?使用了2个Consumer连续处理数据,
而IntStream提供了这种流式动力。从Stream中每取出一个数字,先给c1处理
然后再交给c2处理。所以,以上代码的输出是:
\begin{lstlisting}
	双数=>0
	单数->1
	双数=>2
	单数->3
	双数=>4
	单数->5
	双数=>6
	单数->7
	双数=>8
	单数->9
\end{lstlisting}

\emph{2.读写文件}
\begin{lstlisting}[language=java]
	Path path = Paths.get("/Users/simbaba",
												"Desktop",
												"test.txt");
	Files.lines(path).forEach(System.out::println);
\end{lstlisting}
使用Files轻松应对大多数的文件操作,借助$\lambda$表达式也非常简单。

\emph{3.找出最大和最小}
\begin{lstlisting}[language=java]
	Integer[] ids = {11, 2, 6, 4, 7, 10, 5};
	OptionalInt min = Stream.of(ids).mapToInt(id->id).min();
	OptionalInt max = Stream.of(ids).mapToInt(id->id).max();
	
	min.ifPresent(System.out::println);
	max.ifPresent(System.out::println);
\end{lstlisting}
需要注意Integer[]和int[]的区别,同学们可以尝试改成int[]怎么才能获得结果,
又为什么出现这种问题呢?

\section{数学推理}
$\lambda$表达式,对数学是非常友好的,不必再写一堆函数或者分支语句,
可以写出更自然的代码。

\begin{tikzpicture}[scale=0.5]
  \draw[very thin,color=gray] (-2,-2) grid (12,5);
  \draw[->] (-0.2,0) -- (11.4,0) node[right] {$x$};
  \draw[->] (0,-1.2) -- (0,4.4) node[above] {$f(x)$};
  \draw[color=red, domain=0:3]    plot (\x,{2*\x*\x/9}) node[right] {$f(x) =2x^2/9$};
  \draw[color=blue, domain=3:5]   plot (\x,{(\x+1)/2}) node[right] {$f(x) = (x+1)/2$};
	\draw[color=purple, domain=5:7] plot (\x, -\x+8) node[right] {$f(x) = -x+8$};
	\node at(18, 2) {$f(x)=\begin{cases}
		2x^2/9 & \text{x<3}, \\
		(x+1)/2 & \text{[3,5]} \\
		-x+8 & \text{x>=5}
	\end{cases}$};
\end{tikzpicture}

\emph{参考代码:}
\begin{lstlisting}[language=java]
	DoubleUnaryOperator op1 = (d)->2*d*d/9;
	DoubleUnaryOperator op2 = (d)->(d+1)/2;
	DoubleUnaryOperator op3 = (d)->-d+8;
  // 浮点数不建议直接比较大小,此处忽略
	double d = 4.5d;
	if (d > 5) {
			d = op3.applyAsDouble(d);
	} else if (d > 3) {
			d = op2.applyAsDouble(d);
	} else {
			d= op1.applyAsDouble(d);
	}
	System.out.println("d = " + d);
\end{lstlisting}

\section{更加强大的集合}
支持$\lambda$表达式的编程语言,提供的集合操作也都非常强大。
限于篇幅,本节不能全部涵盖,只介绍基本的使用方法。

\subsection{获得stream}
借助接口的default扩展语法,现在所有的Java集合都已经支持数据流stream,用以遍历和操作数据。
这里的stream和文件流(file stream)稍微有些差别,都是对集合的映射。

我们先看数组是如何转换成stream的,分为两种情况:基本类型和对象类型。
处理方式上有一些区别,请注意使用上的差异。
Java是通过泛型实现的数据容器,其中的\emph{T类型}指的是引用类型。
\vspace{0.5cm}
\noindent
以下是int[]和Integer[]的样例代码:
\begin{lstlisting}[language=Java]
	@Test
	public void visitIntList() {
			int[] intList = {1,2,3,4,5};
			IntStream stream = IntStream.of(intList);
			int[] array = stream.map(n -> n * 2).toArray();
			assertArrayEquals(array, new int[]{2,4,6,8,10});
	}

	@Test
	public void visitIntegerList() {
			Integer[] intList = {1,2,3,4,5};
			IntStream stream = Arrays.stream(intList).mapToInt(n->n);
			int[] array = stream.map(n -> n * 2).toArray();
			assertArrayEquals(array, new int[]{2,4,6,8,10});
	}
\end{lstlisting}

与List不一样,Map相当于有2个集合:keySet, values。
遍历数据的时候,根据需要把keySet或values转换为stream。
使用示例如下:
\begin{lstlisting}[language=Java]
  @Test
	public void map2Stream() {
			HashMap<Integer, String> n2ch = new HashMap<>();
			n2ch.put(1, "abcdefg");
			n2ch.put(2, "hijklmn");
			n2ch.put(3, "opqrst");

			Stream<Integer> keyStream = n2ch.keySet().stream();
			Object[] even = keyStream.filter(key -> key % 2 == 0).map(n2ch::get).toArray();

			assertArrayEquals(even, new String[]{"hijklmn"});
	}
\end{lstlisting}

上述代码,遍历keySet的时候先进行一次过滤,然后再进行一次map操作。
通过map操作又把key遍历转换为values遍历,也就是经过map()操作之后,
stream<Integer>已经转换为了stream<String>。
而n2ch::get语法,是一种lambda表达式借用代码和省略参数的写法。

\subsection{统计功能}
仅把stream当作for循环的替代品,只是用于遍历就有点大材小用了。
对于函数式编程,比较有特色的一个操作\emph{reduce}在此表现地淋漓尽致。
\begin{lstlisting}[language=Java]
	int[] intList = {1,2,3,4,5};
	IntStream stream = IntStream.of(intList);

	OptionalInt optSum = stream.reduce((a, b)->a+b);

	int sum = optSum.getAsInt();
	assertEquals(sum, 1+2+3+4+5);
\end{lstlisting}

使用stream还可以对数据进行分类集约,本书不作详细解释,仅提供演示代码。
如下代码,根据性别把学生分为两类。

\begin{lstlisting}[language=Java]
	Student jim = new Student("jim", 1);
	Student jane = new Student("jane", 0);
	Student tom = new Student("tom", 1);
	Student sunny = new Student("sunny", 0);

	Student[] students = new Student[] {jim, jane, tom, sunny};

	Map<Integer, List<Student>> group = Stream.of(students).collect(
									Collectors.groupingBy(s -> s.sex));

	List female = group.get(1);
	Student[] dst = new Student[]{jim, tom};
	assertArrayEquals(female.toArray(), dst);
\end{lstlisting}

对于数据排序、求最大值/最小值和平均值,使用stream处理也都很简单直观。

\subsection{并行处理}
也可以把stream分成多个子流,同时在多个线程并发处理,充分发挥系统多核硬件性能。
将一个顺序执行的流转变成一个并发的流只需要调用parallel()方法。
先用不同的线程分别处理,最后再合并每个数据流的计算结果。

\begin{lstlisting}[language=Java]
	@Test
	public void parallelSum() {
			int[] nums = {1,2,3,4,5,6,7,8,9,10};

			OptionalInt sum = IntStream.of(nums)
							.parallel()
							.reduce((a, b)->a+b);
			assertEquals(sum.getAsInt(), 55);
	}
\end{lstlisting}

\subsection{无限流}
无论是List还是Map,所能提供的数据都是有限的,并不是取之不竭。
使用stream可以实现一种随取随提供的数据流。典型的应用场景,如随机数集合,每次都可以取不同的数据。
如下代码,产生1000~9999之间的数据,取10个数据出来打印。

\begin{lstlisting}[language=Java]
	Random random = new Random(System.currentTimeMillis());
	IntStream stream = random.ints(1000, 9999);
	stream.limit(10).forEach(System.out::println);
\end{lstlisting}

\section{本章小结}
很显然,使用lambda表达式和stream可以极大地简化程序。
不过这不是JDK8语法糖,确实通过语法和语义手段实现了函数式编程。
值得关注的是,lambda表达式的上下文的依赖值得你去思考。
还有很多的知识不能详尽列举,请自行查阅相关资料。