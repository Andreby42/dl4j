\chapter{无监督学习}
\label{chap:unsupervised}
机器学习主要分为三类:监督学习(supervised Learning)、增强学习(reinforcement learning)、无监督学习(unsupervised
learning)。也存在半监督学习(semi-supervised learning)这种情况,但在此不予讨论。

简单来说,区分监督学习和无监督学习的方法就是输入数据是否有标签(label)。
如若没有标签则为无监督学习。
我们以机器学习中的分类(classfication)算法来举例,
对于分类算法来说,输入的训练数据有特征(feature),有标签(label)。
所谓的学习,其本质就是找到特征和标签间的关系(mapping)。
而去判断一个分类算法的成功与否的方法,便是当我们输入有特征无标签的未知数据时,
能否通过已知关系(参数)得到未知数据标签。

举个简单的例子,有监督学习相当于刷算法题的时候你知道答案,
而无监督学习相当于你根本就没有答案,只能靠自己摸索。
所以,与我们的常识相符,无监督学习的准确度往往比起有监督学习要低得多。
那也许你会问了,既然无监督学习准确度那么低,为什么我们还要用它呢?
那是因为在实际情况下,标签的获取需要极大的人工量,还不能保证变迁的准确性。
所以,无监督学习也是十分重要的。

\section{聚类算法}
聚类是数据挖掘中的概念,其定义为按照某个标准把一个数据集分隔成不同的类或簇,
使得相同类里的数据相似程度尽可能大。反之,不在同一个类里的数据,
其差异也要尽可能大。“物以类聚,人以群分”就能大概指代聚类算法的意思。

讲到这里,我们需要区分聚类(clustering)与分类(classification)之间的区别。
对于聚类来说,我们在聚类时,并不关心其中某一类是什么,我们需要实现的目标只是把
相似的东西分到一起。因此,对于聚类算法来说,只要知道相似度衡量的标准就可以开始训练了。
与之相对的,分类算法需要你告知它,它需要将数据分成哪些具体的类别,
所以分类算法需要从训练集中进行学习,从而明白如何对未知数据进行分类。
这就是聚类和分类的区别。

聚类的步骤主要如下:
1.数据准备:将数据转换为标准输入形式,使用特征标准化等方法;
2.特征选择:从最初的特征中选择最有效的特征,并将其存储于向量中;
3.特征提取:通过对所选择的特征进行转换,得到新的重点特征;
4.聚类:首先选择合适特征类型的某种距离函数进行接近程度的度量,而后执行聚类或分组;
5.聚类结果评估:对聚类结果进行评估,评估主要分为3种:外部有效性评估、内部有效性评估和相关性测试评估。

聚类主要分为层次化聚类算法,划分式聚类算法,基于密度的聚类算法,基于网格的聚类算法,基于模型的聚类算法等。
下面我们就挑出几个重要的算法为大家进行讲解。

\subsection{K-means算法}
K-Means算法的特点是类别的个数是人为给定的,
其假设数据之间的相似度可以使用欧氏距离度量,
如果不能使用欧氏距离度量,要先把数据转换到能用欧氏距离度量。

接下来,我们简单介绍一下流程:
首先,我们有在n维向量当中的一堆点,这里以二维空间为例。

%图片

接着我们随机生成k个聚类中心点,相当于将其分为几个类别。
然后分别计算每一个数据点到这些中心的距离,把距离最短的那个当成自己的类别。
这样就可以发现每个点都已经被分类了(有一个中心点),但是并不准确。

%图片

接下来就是无监督学习,使得分类变得更加准确的时候。
我们一开始随机确定的分类点,这时候就要变化了。
而它变化的标准就是“收复”附近的点,所以它将往所有它这一类别的点的坐标平均值移动,
也就是移向中心。而到达中心后,将再一次判断各个点到k个中心点的距离,
选取离每个点最近的中心点作为它的类别,以此类推。

%图片

伪代码流程如下所示:
\begin{lstlisting}[language=Java]
public static double K-Means(输入数据,中心点个数K){
    获取输入数据的维度Dim和个数N
    随机生成K个Dim维的点
    while(算法未收敛){
        对N个点:计算每个点属于哪一类。
        对于K个中心点:
            1,找出所有属于自己这一类的所有数据点
            2,把自己的坐标修改为这些数据点的中心点坐标
    }
    return 结果
}
\end{lstlisting}

接下来,我们来说明一下k-means算法使用过程中有可能会遇到的问题。
1.测量距离的方法
并非一定要使用欧氏空间这个方法,只需满足以下条件都可以用:
首先有个分类两个点的方法的算符记作$$<\vec{a}, \vec{b}>$$,
并且其具有交换性$<\vec{a}, \vec{b}>=<\vec{b}, \vec{a}>$。
其次需要可以求一堆点的平均值的算法(求中心点):
$\vec{\mu}=\operatorname{Mean}\left(\overrightarrow{a_{1}}, \ldots, \overrightarrow{a_{n}}\right)$
求出后只需满足:$\sum_{i=1}^{n}\left(\vec{\mu}-\overrightarrow{a_{i}}\right)^{2}$。

2.如何知道是否收敛?
使用代价函数:$\tilde{J}=\sum_{i=1}^{C} \sum_{j=1}^{N} r_{i j} \times \nu\left(x_{j}, \mu_{i}\right)$。
其中:$\nu\left(x_{j}, \mu_{i}\right)=\left\|x_{j}-\mu_{i}\right\|^{2}$。
代价函数的差分值小于一定数值的时候(N次越不过最小值点)即可认为是收敛了。

3.代价函数不收敛,怎么办?
首先说一下什么时候容易发生震荡:
在数据点个数比较少而且比较稀疏的时候容易发生这种事情,发生的原因大约有两种常见的:
1、陷入某个环里,然后开始震荡,它将会绕着中心点进行低频振荡。
2、两个点互相交换,每次交换不改变J的值就收敛了,
如果交换以后不幸影响了其它的点,就出现了高频振荡。
这个时候给出一种简单的解决方案:阻尼。
简而言之,就是更新自己位置的时候考虑一下原来的位置,
一般阻尼比(在0~1之间取值)决定收敛速度,收敛的慢了也就不容易震荡,
也就越容易陷入局部极小值,也就是说,不震荡的情况下我们应该把阻尼比尽可能取小一点
$\vec{C}^{u p d}=\vec{C}^{n e w} \times(1-\xi)+\vec{C}^{o l d} \times \xi$
$\vec{C}^{u p d}$是最后中心点的取值,$\vec{C}^{n e w}$是当前集合的中心点,
$\vec{C}^{o l d}$是原来的中心点坐标。

\subsection{DBScan算法}

\section{DL4J示例}